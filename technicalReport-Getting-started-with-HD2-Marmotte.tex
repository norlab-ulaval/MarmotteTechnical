\documentclass[12pt,letterpaper,oneside]{article}

\input{preamble}

%==============================================================
% FILL THIS SECTION

\newcommand{\reportTitle}{Getting started with HD2 - Marmotte}
\newcommand{\reportAuthors}{Nicolas Antonucci}
\newcommand{\reportDate}{\today} % or manually: November 23, 2017

\newcommand{\reportVersions}{
0.1 & \today & Initial writing %\\
%1.0 & November 23, 2017 & Final writing
}
% Change to your specific bibliography file
\addbibresource{./latexGoodPractices/exampleReferences.bib}
%\addbibresource{./references.bib}
%==============================================================




% ---------------------------------------------------------------
% Load style
\input{technicalReportStyle.tex}
% ---------------------------------------------------------------


% Add your other packages here
% make sure to check prembule.tex first!



%================================================================
\begin{document}
\makeCustomTitle
\thispagestyle{titlePage}

% ---------------------------------------------------------------
\begin{abstract}
This is a short technical report to help me get familiar with the project and have all the important information and specifications relative to my task in one place.
\end{abstract}

% ---------------------------------------------------------------
\section{HD2 Treaded Tank Robot Platform}

\begin{figure}[h]
    \centering
    \includegraphics[width=0.4\textwidth]{./figures/platform.jpg}
    \caption{Robot platform}
    \label{fig:my_label}
\end{figure}

\begin{itemize}
    \item From superdroidrobots.com
    \item Can climb obstacles, ascend stairs and drive over most terrain
    \item Controled through Roboclaw motor controller
    \item We use 4 IG52-04 24VDC 285 RPM Gear Motor, 2 of them with encoders. Encoders are not needed for the other 2 as they are connected in line with the same track
\end{itemize}

\begin{figure}[h]
    \centering
    \includegraphics[width=0.8\textwidth]{figures/motorSpecs.jpg}
    \caption{Motor config}
    \label{fig:my_label}
\end{figure}

% ---------------------------------------------------------------
\newpage
\section{D.C. Geared motors}%

\begin{itemize}
    \item Name: IG52-04 24VDC 285 RPM Gear Motor
    \item Variable speed and reversible
    \item Dual channel quadrature encoder
    \item \textbf{Each channel requires a 1k pull up resistor to Vcc}
\end{itemize}

\begin{tabular}{ c c }
    Motor & Pull up resistor attached \\
    %\hline
     \raisebox{-\totalheight}{\includegraphics[width=0.45\textwidth]{figures/motor.jpg}} & \raisebox{-\totalheight}{\includegraphics[width=0.45\textwidth]{figures/gearMotorPullUpBoard.jpg}} \\
\end{tabular}

\begin{tabular}{ c c }
    Cable layout & Schematic \\
    %\hline
     \raisebox{-\totalheight}{\includegraphics[width=0.45\textwidth]{figures/encoderPins.jpg}} & \raisebox{-\totalheight}{\includegraphics[width=0.45\textwidth]{figures/encoderPins2.jpg}} \\
\end{tabular}

\begin{tabular}{ c }
    Electrical caracteristics \\
    %\hline
     \raisebox{-\totalheight}{\includegraphics[width=0.9\textwidth]{figures/electricalCaracteristics.jpg}} \\
\end{tabular}

\begin{itemize}
    \item To reduce noise as much as possible in the system, twist positive and negative wires together and use ferrite beads at each motor connection.
    \item Bigger ferrite first over both wires and one smaller ferrite bead over each wire. (Don't forget the heatshrink)
\end{itemize}


%Here some examples of references \cite{Pomerleau2013,Pomerleau2014}.

% ---------------------------------------------------------------
\newpage
\section{Roboclaw 2x30A Motor Controller}%

\begin{figure}[h]
    \centering
    \includegraphics[width=0.3\textwidth]{figures/roboclaw.jpg}
    \caption{Motor Controller}
    \label{fig:my_label}
\end{figure}

\begin{tabular}{ c c }
    Hardware overview & Control interface \\
    %\hline
     \raisebox{-\totalheight}{\includegraphics[width=0.30\textwidth]{figures/roboclawHardwareOverview.jpg}} & \raisebox{-\totalheight}{\includegraphics[width=0.55\textwidth]{figures/roboclawPinLayout.jpg}} \\
\end{tabular}

\begin{itemize}
    \item Connect batteries to the main terminal (G).
    \item Connect encoder through pull up resistor chip to pins in section D.
    \item See schematic below for details.
    \item For initial tests connect roboclaw through usb to computer (batteries have to be connected to roboclaw to function). Use motion studio to check if motors and sensors work.
\end{itemize}

\newpage
\begin{figure}[h]
    \centering
    \includegraphics[width=0.9\textwidth]{figures/roboclawHardwareTable.jpg}
    \caption{Hardware overview reference table}
    \label{fig:my_label}
\end{figure}

\begin{figure}[h]
    \centering
    \includegraphics[width=0.9\textwidth]{figures/roboclawPinTable.jpg}
    \caption{Control interface reference table}
    \label{fig:my_label}
\end{figure}

\newpage
\begin{figure}[h]
    \centering
    \includegraphics[width=0.9\textwidth]{figures/electricalSpecsTable.jpg}
    \caption{Electrical specifications}
    \label{fig:my_label}
\end{figure}

\newpage
\begin{figure}[h]
    \centering
    \includegraphics[width=0.9\textwidth]{figures/wiring.jpg}
    \caption{Safety wiring}
    \label{fig:my_label}
\end{figure}

% ---------------------------------------------------------------
\section{Motion studio}

% ---------------------------------------------------------------
\section{ROS Driver}

% ---------------------------------------------------------------
\section{Remote Control}

% ---------------------------------------------------------------
\printbibliography

\end{document}